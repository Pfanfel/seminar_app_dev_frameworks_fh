\chapter{Einleitung}

Weltweit wurden im Jahr 2019 1.540.655 Smartphones verkauft. Diese Anzahl an verkauften Endgeräten folgt einem seit 2016 stabilen Trend von c.a. 1.500.000 vertriebenen Exemplaren jährlich. Aufgrund der weltweiten Covid-19 Pandemie sank die Anzahl an abgesetzten Smartphones global um 10.5\% auf 1.378.719 Einheiten, jedoch wird für das Jahr 2021 prognostiziert, dass sich der Markt erholt\cite{statista2_sales}\cite{gartner_sales}. In relativen Zahlen bedeutet dies, dass derzeit etwa 48\% der Weltbevölkerung ein Smartphone besitzen. Im Verglich dazu betrug der Anteil der Smartphonebesitzer im Jahr 2016 33,5\%\cite{bankmycell}.\\

Daraus ergibt sich, dass Smartphones sich hoher Beliebtheit erfreuen und inzwischen fest in unser alltägliches Leben integriert sind. Dies kann damit begründet werden, dass die Möglichkeit besteht eine Vielzahl verschiedener Apps zu installieren mit deren Hilfe das Gerät um eine Vielfalt an Funktionalitäten erweitert wird, die den Bedürfnissen und Wünschen des Nutzers entsprechen. Die verschiedenen Smartphones nutzen unterschiedliche Betriebssysteme, sodass eine App für die breite Masse mit den verschiedenen Plattformen kompatibel sein sollte, um eine möglichst große Anzahl an Nutzern zu erreichen.\\

Android und iOS stellen 99\% des Marktanteils der mobilen Betriebssysteme dar. Dabei ist Googles Android der Marktführer mit 72\% weltweit und iOS, Apples proprietäres Betriebssystem, mit ca. 27\% auf dem zweiten Platz \cite{stat1}. Somit stellen beide Betriebssysteme eine hinreichend große Verteilung dar, um im Folgenden berücksichtigt zu werden.\\

Bei Betrachtung der Relevanz von Betriebssystemversionen, fällt auf, dass Android Version 8 und höher auf 82\% aller Geräte zum Einsatz kommt. Im iOS Ökosystem ist die älteste relevante Version iOS 12.4 mit c.a. 7\% Anteil\cite{stat3}\cite{stat4}.\\

\newpage

Die Verteilung der Betriebssysteme und deren verwendete Versionen spielen bei der Evaluation des Entwicklungssatzes eine wichtige Rolle. Aufgrund der Tatsache, dass nicht alle Nutzer sofort das Betriebssystem aktualisieren, entsteht die Notwendigkeit eine mobile App auf den am meisten verwendeten Betriebssystemversionen explizit zu testen. Zudem müssen teilweise mehrere Versionen derselben App gepflegt werden, um ein breites Spektrum der Betriebssystemversionen zu unterstützen. Dieser Support verbraucht wertvolle Entwicklerressourcen, sodass eine Abwägung der relevanten Versionen notwendig wird\cite{flutter_support}. Ebenfalls relevant für einige Entwicklungsansätze sind die verschiedenen Web-Browser wie Firefox, Chrome oder Safari und deren Versionen. Auf die Eigenschaften und Unterschiede der verschiedenen Web-Browser wird jedoch im Laufe dieser Arbeit nicht näher eingegangen, da diese vor allem bei webbasierten Entwicklungsansätzen relevant sind, welche nur einen Teil dieser Seminararbeit ausmachen.\\

Ziel dieses Seminars ist es einen konzeptionellen Überblick über die derzeit vorherrschenden Mobile App Framework Technologien zu geben und auf die sich aktuell im Trend befindlichen näher einzugehen.
Es soll sowohl ein rudimentäres Verständnis über die verschiedenen Technologien als auch eine Basis geschaffen werden, auf welcher aufbauend Entscheidungen für oder gegen einen gewissen Ansatz getroffen werden können soll.\\

